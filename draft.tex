\documentclass[11pt]{amsart}
\usepackage[letterpaper,left=125pt,right=125pt]{geometry}

\usepackage[english]{babel}
\usepackage{amsmath, amssymb, amsfonts, amsthm, graphicx}

\usepackage[colorlinks,linkcolor=blue,citecolor=blue,urlcolor=blue]{hyperref}
\usepackage[backend=biber,isbn=false,url=false]{biblatex} 
\addbibresource{sample.bib}

\newtheorem{thm}{Theorem}[section]
\newtheorem{lem}[thm]{Lemma}
\newtheorem{cor}[thm]{Corollary}
\newtheorem{prop}[thm]{Proposition}
\newtheorem{alg}[thm]{Algorithm}

\theoremstyle{definition}
\newtheorem{defn}[thm]{Definition}
\newtheorem{conj}[thm]{Conjecture}

\theoremstyle{example}
\newtheorem{prob}[thm]{Problem}
\newtheorem{que}[thm]{Question}

\theoremstyle{remark}
\newtheorem{exmp}[thm]{Example}
\newtheorem{rem}[thm]{Remark}
\newtheorem{claim}[thm]{Claim}  
\renewcommand{\theclaim}{}

\numberwithin{equation}{section}

\usepackage[usenames,dvipsnames]{xcolor}
\usepackage{enumitem}
\setlist{itemsep=.5em, parsep = .5em}
\usepackage{lipsum}

\usepackage{tcolorbox}
\tcbuselibrary{most}

\tcolorboxenvironment{thm}{ enhanced jigsaw, colframe=Aquamarine, colback=Aquamarine!8!white, breakable, before skip=5pt plus 10pt minus 3pt, after skip=5pt plus 10pt minus 3pt, grow to left by=8pt, left*=0pt, grow to right by=5pt, right*=0pt, top=0pt, bottom=0pt, sharp corners, rightrule=0pt, toprule=0pt, bottomrule=0pt, leftrule=3pt, parbox=false}

\tcolorboxenvironment{prop}{ enhanced jigsaw, colframe=cyan, colback=cyan!8!white, breakable, before skip=5pt plus 10pt minus 3pt, after skip=5pt plus 10pt minus 3pt, grow to left by=8pt, left*=0pt, grow to right by=5pt, right*=0pt, top=0pt, bottom=0pt, sharp corners, rightrule=0pt, toprule=0pt, bottomrule=0pt, leftrule=3pt, parbox=false}

\tcolorboxenvironment{defn}{ enhanced jigsaw, colframe=OliveGreen, colback=OliveGreen!8!white, breakable, before skip=5pt plus 10pt minus 3pt, after skip=5pt plus 10pt minus 3pt, grow to left by=8pt, left*=0pt, grow to right by=5pt, right*=0pt, top=0pt, bottom=0pt, sharp corners, rightrule=0pt, toprule=0pt, bottomrule=0pt, leftrule=3pt, parbox=false}

\tcolorboxenvironment{proof}{ enhanced jigsaw, colframe=gray, colback=gray!8!white, breakable, before skip=5pt plus 10pt minus 3pt, after skip=5pt plus 10pt minus 3pt, grow to left by=8pt, left*=0pt, grow to right by=5pt, right*=0pt, top=0pt, bottom=0pt, sharp corners, rightrule=0pt, toprule=0pt, bottomrule=0pt, leftrule=3pt, parbox=false}

\DeclareMathOperator{\dist}{dist}
\DeclareMathOperator{\Ima}{Im}
\DeclareMathOperator{\dimension}{dim}
\DeclareMathOperator{\Supp}{Supp}

\title[Codes from Algebraic Geometry]{Bounds on Coding Theory from Algebraic Geometry}
\author{Guilherme Zeus Dantas e Moura}
\date{last compiled \today} % <---- Change this date to the format YYYY-MM-DD

\begin{document}

%\begin{abstract} % NOT REQUIRED
%    Coding theory is concerned with finding efficient ways to encode a message so that one may correct errors in the message. In algebraic coding theory, we study efficient codes generated from algebraic geometric methods.
%    
%    In this paper, I will construct the Reed--Solomon codes, generalize them using projective curves, and understand the results from \cite{TVZ82} on finding a bound better than the well-known Gilbert--Varshamov one.
%\end{abstract} 

\maketitle

\section{Coding theory} \label{s:codingtheory}

WRITE INTRODUCTION WITH ALICE AND BOB.

\begin{defn}[Code]
    A \emph{code $C$ over an alphabet $A$} is a subset of $A^n = A \times \cdots \times A$.
    We define $n$ as the \emph{length of $C$}.
    A code $C$ over a field $A$ is a \emph{linear code} if $C$ is a vector subspace of $A^n$. An element of a code $C$ is called a \emph{code word}.
\end{defn}

In this paper, $A$ is a finite field unless otherwise stated.

\begin{defn}[Hamming distance]
    We define \emph{Hamming distance} between $\mathbf{x} = (x_1, \dots, x_n), \mathbf{y} = (y_1, \dots, y_n) \in A^n$ as
    \begin{equation}
		\dist(\mathbf{x}, \mathbf{y}) = \#\left( x_i \neq y_i \mid i \in \{1, 2, \dots, n\} \right),
	\end{equation}
    in other words, the number of positions $\mathbf{x}$ and $\mathbf{y}$ differ.
\end{defn}

\begin{prop}
    Hamming distance is a metric over $A^n$, i.e., the following holds for any $\mathbf{x}, \mathbf{y}, \mathbf{z} \in A^n$:
    \begin{itemize}[noitemsep]
        \item $\dist(\mathbf{x}, \mathbf{y}) = 0 \iff \mathbf{x} = \mathbf{y}$;
        \item $\dist(\mathbf{x}, \mathbf{y}) = \dist(\mathbf{x}, \mathbf{y})$;
        \item $\dist(\mathbf{x}, \mathbf{y}) \le \dist(\mathbf{x}, \mathbf{z}) + \dist(\mathbf{z}, \mathbf{y})$.
    \end{itemize}
\end{prop}

\begin{defn}[Parameters of a code]
    If $C$ is a linear code over $A$, we define \emph{dimension of $C$} as $k = \dimension_A(C)$ and \emph{minimum distance of $C$} as $d = \min\left\{\dist(\mathbf{x}, \mathbf{y}) \mid \mathbf{x}, \mathbf{y} \in C\right\}$. (\emph{If $C$ is a nonlinear code over an alphabet with size $q$, we can coherently define $k = \log_q{\left|C\right|}$.}) The length $n$, dimension $k$ and minimum distance $d$ are the \emph{parameters} of $C$.
\end{defn}

Suppose Alice wants to send a message to Bob through a noisy channel. They previously agree on a choice of code $C \subset A^n$, with parameters $n, k, d$. Alice will choose one of the $|A|^k$ code words and send it to Bob. Since the channel is not a perfect medium, some positions of the code may change; however, if less than $\frac{d}{2}$ of such changes occur, Bob can take the closest code word to the receiving message using Hamming distance and restore the original message.

Thus, a good code has two properties: it has large $d$ with respect to $n$, in order to correct as many errors as possible; but also has large $k$ with respect to $n$, so that Alice has a wider variety of possible messages to send and send more information.

\begin{defn}
    If $C$ is a code, its code rate is $R = k/n$ and its relative minimum distance is $\delta = d/n$. Note that $R, \delta \in [0, 1]$.
\end{defn}

Therefore, a good code is one with large $R$ --- not much redundancy --- and large $\delta$ --- corrects many errors. 

\section{Singleton bound and a promising example} \label{s:singleton}
\begin{thm}[Singleton Bound]\label{thm:singleton_bound}
If $C$ is a code with parameters $n, k, d$, then 
	\begin{equation}
		k + d \le n + 1,
	\end{equation} 
	or equivalently, 
	\begin{equation}
		R + \delta \le 1 + 1/n.
	\end{equation}
\end{thm}

\begin{proof}
	We will provide the proof for Theorem \ref{thm:singleton_bound} when $C$ is a linear code.
	
	WRITE PROOF.
\end{proof}

\begin{defn}[Reed--Solomon Codes]\label{defn:rs_codes}
	Let $q$ be a power of a prime, and $\mathbb{F}_q = \{\alpha_1, \alpha_2, \dots, \alpha_q\}$ the field with $q$ elements. Let $k$ be an integer, and ${L}_k$ the set of all polynomials over $\mathbb{F}_q$ with degree smaller than $k$. Let $k \le n \le q$ be an integer. The Reed--Solomon code $RS_q(n, k)$ over $\mathbb{F}_q$ is 
	\begin{equation}
		RS_q(n, k) = \left\{\left(f(\alpha_1), f(\alpha_2), \dots, f(\alpha_n)\right) \mid f \in {L}_k\right\}.
	\end{equation}
\end{defn}

\begin{prop}
	The Reed--Solomon code $R_q(n, k)$ is a linear code with length $n$, dimension $k$ and minimum distance $n - k + 1$.
	Thus, any Reed--Solomon code meets the inequality of the Singleton Bound.
\end{prop}

\begin{proof}
	$R_q(n, k)$ is a subset of $\mathbb{F}_q^n$, thus it has length $n$.  
	Note that $ L_k$ is a vector space over $\mathbb{F}_q$. Note that $\{1, x, x^2, \dots, x^{k-1}\}$ is a choice of basis for this vector space, thus it has dimension $k$. Consider the map $\phi:  L_k \to \mathbb{F}_q^n$ given by 
	\begin{equation}
		f \mapsto (f(\alpha_1), f(\alpha_2), \dots, f(\alpha_n)).
	\end{equation}

	Note that the map $\phi$ is a linear transformation. Thus, its image $\Ima \phi = RS_q(n, k)$ is also a vector space.  
	Additionally, if $\phi(f) = \phi(g)$, then $f - g$ has at least $n$ roots, but has degree less than $n$; thus $f - g$ is the zero polynomial, which implies $f = g$. Therefore, $\phi$ is also injective. This implies that the dimension of the domain $ L_k$ is the same as the dimension of the image $RS_q(n, k)$, i.e.,  $\dimension RS_q(n, k) = k$.
	
	Finally, consider distinct $f, g \in  L_k$ and define $d = \dist(\phi(f), \phi(g))$, $f - g$ has at least $n - d$ roots. Furthermore, $f - g$ is a non-zero polynomial with degree less than $k$, thus has at most $k - 1$ roots. Then,  $ k - 1 \ge \#\ \text{roots} \ge n - d$.
	If we choose $f, g$ such that $d$ is the minimal distance, we get $k + d_{\mathrm{min}} \ge n + 1$, which together with \nameref{thm:singleton_bound} implies 
	\begin{equation}
		k + d_\text{min} = n + 1.
	\end{equation}

\end{proof}

The Reed--Solomon codes are very good codes in the sense that key have the largest possible sum $k + d$ for their length $n$. However, Reed--Solomon codes are limited because their length is at most the alphabet size. So, a question naturally arises: Given fixed $\mathbb{F}_q$, are there codes over $\mathbb{F}_q$ with arbitrarily large $n$ and $R + \delta = 1 + 1/n$? If not, how large can $R$ and $\delta$ be when $n$ gets larger? The Gilbert--Varshamov bound shows that there are codes with 
\begin{equation}
	1 - R \approx q(\delta), \text{as $n \to \infty$,}
\end{equation}
in which
\begin{equation}
	q(x) = x \log_q(q-1) - x\log_q(x) - (1-x)\log_q(x).
\end{equation}
Are there any better codes?

\section{Rational Functions and Divisors}

\begin{defn}[Rational Function]
	A rational function $f$	is a function which is the ratio $g/h$ of two polynomials.
	It is homogeneous if $g, h$ are homogeneous.
	After cancelling common roots of $g, h$, the roots of $g$ are called \emph{zeros} of $f$ and the roots of $h$ are called the \emph{poles} of $f$.
\end{defn}

	We say $f$ has order $n$ in $P$ if $P$ is a zero of muliplicity $n$; order $-n$ if $P$ is a pole with multiplicity $n$; order $0$, otherwise.

\begin{defn}[Divisor] 
	Let $F$ be the algebraic closure of $\mathbb{F}_q$.
	Let $X$ be an irreducible nonsingular projective curve in $N$-dimensional projective space over $F$.
	A \emph{divisor} on $X$ is a formal finite sum of the form $D = \sum a_P P$, where $P$ are points of $X$, $a_P$ are integers and $a_P = 0$ for all but finitely many points $P$.
	The \emph{degree} of $D$ is $\sum n_P$.
	The \emph{support} $\Supp D$ is the set $\{P \in X : a_P \neq 0\}$
\end{defn}

If $D = \sum n_p P$, then define the vector space $\mathcal L(D)$ as the set of all homogeneous rational functions $f$ such that the order of $f$ at each point $P$ of $X$ is greater or equal to $n_P$.
For our study, an important theorem is the following:
\begin{thm}[Riemman--Roch Theorem, \cite{Wal00}]
	Let $X$ be a nonsingular projective curve of genus\footnote{For the reader that is not familiar with genus, it is enough to know that it is an integer that can be calculated for any given curve.} $g$ defined over the field $\mathbb{F}_q$ and let $D$ be a divisor on $X$. Then
	\begin{equation}
		\dim \mathcal L(D) \ge \deg D + 1 - g,
	\end{equation}
	with equality holding if $\deg D > 2g - 2$.
\end{thm}

%Note that, if the degree of $D$ is negative, then
%\begin{equation}
%	\mathcal L(D) = \{0\}.
%\end{equation}
%Additionally, the \emph{Riemann--Rock theorem} implies that the dimension of the vector space $\mathcal L(D)$ is 
%\begin{equation}
%	 \dim\mathcal L(D) \ge \deg D - g + 1,
%\end{equation}
%where $g$ denotes the genus of $X$ (for the reader who is not familiar with the term genus, it suffices to know that it is an integer which can be calculated for any given curve).

\section{Generalized Reed--Solomon codes} \label{s:grs}
Let $\mathbb{P}^1(\mathbb{F}_q)$ denote the projective line over $\mathbb{F}$. We will write $(a : b)$ to denote the projective point corresponding to the $1$-dimensional vector space through $(a, b)$. The points on $\mathbb{P}^1(\mathbb{F}_q)$ are the points
\begin{equation}
	P_i = (\alpha_i : 1), \qquad 1 \le i \le q,
\end{equation}
and
\begin{equation}
	P_\infty = (1 : 0).
\end{equation}

Following \cite{LS87}, let $\mathcal L_k$ be the set of two-variable homogeneous rational functions which have a pole of order less than $k$ in the point $Q$.

\begin{prop}
	The sets $\mathcal L_k$ and $L_k$ are mapped with a bijection $\phi: f(x) \mapsto f(x/y)$.
\end{prop}
\begin{proof}
	WRITE PROOF.
\end{proof}

Then, we can rewrite the Reed--Solomon code from \ref{defn:rs_codes} as
\begin{equation}
	RS_q(n, k) = \{f(P_1), f(P_2), \dots, f(P_n) \mid f \in \mathcal L_k\}.
\end{equation}



\begin{defn}[Algebraic Geometric Code]
\end{defn}

We want large $R$ and $\delta$, and these codes yield 
	\begin{equation}
		\label{eq:g/n} R + \delta \ge 1 + 1/n - g/n,
	\end{equation}
where $n$ is the number of rational points of a curve $X$, with genus $g$.

\section{Final thoughts} \label{s:tvz} On equation \eqref{eq:g/n}, we observe that good algebraic geometric codes are generated by curves with a large ratio between $n$ and $g$. On \cite{TVZ82}, the authors present a sequence of such curves, with $n/g$ large enough to create a better bound than the Gilbert--Varshamov one.

\printbibliography

\end{document}

\documentclass[11pt, oneside]{amsart}
\usepackage[letterpaper,left=125pt,right=125pt]{geometry}

\usepackage[english]{babel}
\usepackage{amsmath, amssymb, amsfonts, amsthm, graphicx}
\usepackage[mathscr]{eucal}
\usepackage[symbol, perpage]{footmisc}

\usepackage[colorlinks,linkcolor=blue,citecolor=blue,urlcolor=blue]{hyperref}
\usepackage[backend=biber,isbn=false,url=false]{biblatex} 
\addbibresource{sample.bib}

\newtheorem{thm}{Theorem}[section]
\newtheorem{lem}[thm]{Lemma}
\newtheorem{cor}[thm]{Corollary}
\newtheorem{prop}[thm]{Proposition}

\theoremstyle{definition}
\newtheorem{defn}[thm]{Definition}

\theoremstyle{remark}
\newtheorem{rem}[thm]{Remark}
\newtheorem*{exmp*}{Example}

\numberwithin{equation}{section}

\usepackage[usenames,dvipsnames]{xcolor}
\usepackage{enumitem}
\setlist{itemsep=.5em, parsep = .5em}
\usepackage{lipsum}

\usepackage{tcolorbox}
\tcbuselibrary{most}

%\tcolorboxenvironment{thm}{ enhanced jigsaw, colframe=Aquamarine, colback=Aquamarine!8!white, breakable, before skip=5pt plus 10pt minus 3pt, after skip=5pt plus 10pt minus 3pt, grow to left by=8pt, left*=0pt, grow to right by=5pt, right*=0pt, top=0pt, bottom=0pt, sharp corners, rightrule=0pt, toprule=0pt, bottomrule=0pt, leftrule=3pt, parbox=false}

%\tcolorboxenvironment{prop}{ enhanced jigsaw, colframe=cyan, colback=cyan!8!white, breakable, before skip=5pt plus 10pt minus 3pt, after skip=5pt plus 10pt minus 3pt, grow to left by=8pt, left*=0pt, grow to right by=5pt, right*=0pt, top=0pt, bottom=0pt, sharp corners, rightrule=0pt, toprule=0pt, bottomrule=0pt, leftrule=3pt, parbox=false}

%\tcolorboxenvironment{defn}{ enhanced jigsaw, colframe=OliveGreen, colback=OliveGreen!8!white, breakable, before skip=5pt plus 10pt minus 3pt, after skip=5pt plus 10pt minus 3pt, grow to left by=8pt, left*=0pt, grow to right by=5pt, right*=0pt, top=0pt, bottom=0pt, sharp corners, rightrule=0pt, toprule=0pt, bottomrule=0pt, leftrule=3pt, parbox=false}

%\tcolorboxenvironment{proof}{ enhanced jigsaw, colframe=gray, colback=gray!8!white, breakable, before skip=5pt plus 10pt minus 3pt, after skip=5pt plus 10pt minus 3pt, grow to left by=8pt, left*=0pt, grow to right by=5pt, right*=0pt, top=0pt, bottom=0pt, sharp corners, rightrule=0pt, toprule=0pt, bottomrule=0pt, leftrule=3pt, parbox=false}

\DeclareMathOperator{\dist}{dist}
\DeclareMathOperator{\Ima}{Im}
\DeclareMathOperator{\dimension}{dim}
\DeclareMathOperator{\Supp}{Supp}

\usepackage{todonotes}
\setuptodonotes{size=\tiny, linecolor=black, backgroundcolor=yellow!30}

\usepackage{makecell}

\title[Codes from Algebraic Geometry]{Bounds on Coding Theory\\from Algebraic Geometry}
\author{Guilherme Zeus Dantas e Moura}
\date{2021-05-07} % <---- Change this date to the format YYYY-MM-DD

\begin{document}

%\begin{abstract} % NOT REQUIRED
%	Coding theory is concerned with finding efficient ways to encode a message so that one may correct errors in the message.
%	In algebraic coding theory, we study efficient codes generated from algebraic geometric methods.
%    
%    In this paper, I will construct the Reed--Solomon codes, generalize them using projective curves, and understand the results from \cite{TVZ82} on finding a bound better than the well-known Gilbert--Varshamov one.
%\end{abstract} 

\maketitle

\section{Coding theory} \label{s:codingtheory}

Suppose Alice wants to send a message to Bob though a noisy channel.
When Bob receives the message, it is possible that some of the information is misinterpreted.
For example, if Alice sends the message `\texttt{food}' through the noisy channel, one of its letters may be misinterpreted and Bob could actually receive `\texttt{\underline{m}ood}'.
So, the question Coding Theory tries to answer is:
How can Alice and Bob agree on a system beforehand so that, if Alice sends a message to Bob, even if some misinterpretations occur, Bob will be able to understand the correct meaning?

\begin{defn}[Code] \label{defn:code}
    A \emph{code $C$ over an alphabet $A$} is a subset of $A^n = A \times \cdots \times A$.
    We define $n$ as the \emph{length of $C$}.
	An element of a code $C$ is called a \emph{code word}.
	A code $C$ over a finite field $\mathbb{F}_q$ is a \emph{linear code} if $C$ is a vector subspace of $\mathbb{F}_q^n$.
\end{defn}

Back to the analogy with Alice and Bob, the code $C$ is the set of all the messages that Alice may send to Bob, according to their agreement.

\begin{defn}[Hamming distance] \label{defn:hamming}
    We define \emph{Hamming distance} between $\mathbf{x} = (x_1, \dots, x_n), \mathbf{y} = (y_1, \dots, y_n) \in A^n$ as
    \begin{equation}
		\dist(\mathbf{x}, \mathbf{y}) = \#\left(i \in \{1, 2, \dots, n\} \mid x_i \neq y_i\right),
	\end{equation}
    in other words, the number of positions $\mathbf{x}$ and $\mathbf{y}$ differ.
\end{defn}

\begin{prop} \label{prop:hamming}
    Hamming distance is a metric over $A^n$, i.e., the following holds for any $\mathbf{x}, \mathbf{y}, \mathbf{z} \in A^n$:
    \begin{itemize}[noitemsep]
        \item $\dist(\mathbf{x}, \mathbf{y}) = 0 \iff \mathbf{x} = \mathbf{y}$;
        \item $\dist(\mathbf{x}, \mathbf{y}) = \dist(\mathbf{y}, \mathbf{x})$;
        \item $\dist(\mathbf{x}, \mathbf{y}) \le \dist(\mathbf{x}, \mathbf{z}) + \dist(\mathbf{z}, \mathbf{y})$.
    \end{itemize}
\end{prop}

The proof of Proposition \ref{prop:hamming} is ommited from this short paper.

\begin{defn}[Parameters of a code]
	If $C$ is a code over $A$, we define \emph{dimension of $C$} as $k = \log_{|A|}|C|$ and \emph{minimum distance of $C$} as $d = \min\left\{\dist(\mathbf{x}, \mathbf{y}) \mid \mathbf{x} \neq \mathbf{y} \in C\right\}$.
	(\emph{If $C$ is a linear code, the definition above is equivalent to $k = \dimension_{A}(C)$.})
	The length $n$, dimension $k$ and minimum distance $d$ are the \emph{parameters} of $C$.
\end{defn}

Alice and Bob agree on a choice of code $C \subset A^n$, with parameters $n, k, d$.
To send a message, Alice will choose one of the $|A|^k$ code words and send it to Bob.
Since the channel is not a perfect medium, some positions of the code may change;
however, if less than $\frac{d}{2}$ of such changes occur, Bob can take the closest code word to the receiving message using Hamming distance and restore the original message.

Thus, a good code has two properties: it has large $d$ with respect to $n$, in order to correct as many errors as possible;
but also has large $k$ with respect to $n$, so that Alice has a wider variety of possible messages to send and send more information.

\begin{defn}
	If $C$ is a code, its \emph{code rate} is $R = k/n$ and its \emph{relative minimum distance} is $\delta = d/n$.
	Note that $R, \delta \in [0, 1]$.
\end{defn}

Therefore, a good code is one with large $R$ --- not much redundancy --- and large $\delta$ --- corrects many errors. 

\section{The singleton bound and a promising example} \label{s:singleton}
\begin{thm}[Singleton Bound]\label{thm:singleton_bound}
If $C$ is a code with parameters $n, k, d$, then 
	\begin{equation} % IF SPACE NEEDED, TRANSFORM THESE EQUATIONS INTO INLINE
		k + d \le n + 1,
	\end{equation} 
	or equivalently, 
	\begin{equation}
		R + \delta \le 1 + 1/n.
	\end{equation}
\end{thm}
	
We will provide the proof for Theorem \ref{thm:singleton_bound} when $C$ is a linear code over a field $K$.

\begin{proof}
	Let $W := \{\mathbf{x} = (x_1, \dots, x_n) \mid x_{d} = \dots = x_n = 0\}$, which is a vector subspace of $K^n$ with dimension $d - 1$. Since $C$ is a vector space, $\vec 0 \in C$. For any non-zero vector $\vec v \in C$, $\dist(\vec 0, \vec v) \ge d$, thus $\vec v$ has at least $d$ non-zero entries, and therefore $\vec v \notin W$. Thus, $W \cap C = \{\vec 0\}$.

	Let $\vec w_1, \dots, \vec w_{d-1}$ and $\vec v_1, \dots, \vec v_k$ be a choice of basis for $W$ and $C$, respectively. Since $W \cap C = \{\vec 0\}$, the vectors $\vec w_1, \dots, \vec w_{d-1}, \vec v_1, \dots, \vec v_k$ are linearly independent. They all are vectors in $K^n$, therefore, $k + d - 1 \le n$.
\end{proof}

\begin{defn}[Reed--Solomon codes, \cite{LS87}]\label{defn:rs_codes}
	Let $q$ be a power of a prime, and $\mathbb{F}_q = \{\alpha_1, \alpha_2, \dots, \alpha_q\}$ the field with $q$ elements.
	Let $k$ be an integer, and ${L}_k$ the set of all polynomials over $\mathbb{F}_q$ with degree smaller than $k$.
	Let $k \le n \le q$ be an integer. The Reed--Solomon code $RS_q(n, k)$ over $\mathbb{F}_q$ is 
	\begin{equation}
		RS_q(n, k) = \left\{\left(f(\alpha_1), f(\alpha_2), \dots, f(\alpha_n)\right) \mid f \in {L}_k\right\}.
	\end{equation}
\end{defn}

\begin{prop}
	The Reed--Solomon code $RS_q(n, k)$ is a linear code with length $n$, dimension $k$ and minimum distance $n - k + 1$.
	Thus, any Reed--Solomon code meets the inequality of the Singleton Bound.
\end{prop}

\begin{proof}
	$RS_q(n, k)$ is a subset of $\mathbb{F}_q^n$, thus it has length $n$.  
	Note that $ \mathcal L_k$ is a vector space over $\mathbb{F}_q$.
	Note that $\{1, x, x^2, \dots, x^{k-1}\}$ is a choice of basis for this vector space, thus it has dimension $k$.
	Consider the map $\phi:  \mathcal L_k \to \mathbb{F}_q^n$ given by 
	\begin{equation}
		f \mapsto (f(\alpha_1), f(\alpha_2), \dots, f(\alpha_n)).
	\end{equation}

	Note that the map $\phi$ is a linear transformation.
	Thus, its image $\Ima \phi = RS_q(n, k)$ is also a vector space.  
	Additionally, if $\phi(f) = \phi(g)$, then $f - g$ has at least $n$ roots, but has degree less than $n$; thus $f - g$ is the zero polynomial, which implies $f = g$.
	Therefore, $\phi$ is also injective.
	This implies that the dimension of the domain $ \mathcal L_k$ is the same as the dimension of the image $RS_q(n, k)$, i.e.,  $\dimension RS_q(n, k) = k$.
	
	Finally, consider distinct $f, g \in  \mathcal L_k$ and define $d = \dist(\phi(f), \phi(g))$, $f - g$ has at least $n - d$ roots.
	Furthermore, $f - g$ is a non-zero polynomial with degree less than $k$, thus has at most $k - 1$ roots.
	Then,  $ k - 1 \ge \#\ \text{roots} \ge n - d$.
	If we choose $f, g$ such that $d$ is the minimal distance, we get $k + d_{\mathrm{min}} \ge n + 1$, which together with the Singleton Bound, implies $k + d_\text{min} = n + 1$.
\end{proof}

The Reed--Solomon codes are very good codes in the sense that they have the largest possible sum $k + d$ for their length $n$.
However, Reed--Solomon codes are limited because their length is at most the alphabet size.
So, a question naturally arises: Given fixed $\mathbb{F}_q$, are there codes over $\mathbb{F}_q$ with arbitrarily large $n$ and $R + \delta = 1 + 1/n$?
If not, how large can $R$ and $\delta$ be when $n$ gets larger?
The well-known Gilbert--Varshamov bound shows that there are codes with 
\begin{equation}
	1 - R \approx H_q(\delta), \text{ as $n \to \infty$,}
\end{equation}
in which
\begin{equation}
	H_q(x) = x \log_q(q-1) - x\log_q(x) - (1-x)\log_q(x).
\end{equation}
The article \cite{TVZ82} uses Algebraic Geometry to show that there are codes that give better bounds.

\section{Curves, rational functions and divisors}

Through this section, let $K$ be a field and $L$ its algebraic closure.

\begin{defn}[Algebraic plane curves]
	An affine algebraic plane curve $\mathfrak C_f$ is the zero set of a polynomial $f \in K[x, y]$.
	A projective algebraic plane curve $\mathfrak C_F$ is the zero set in a projective plane of a homogeneous polynomial $F \in K[X, Y, Z]$.\footnote{In more general terms, algebraic curves (as opposed to algebraic plane curves) are algebraic varieties of dimension $1$. For this paper, we will not worry about this; our discussion will be based on algebraic plane curves, but the results also follow in higher dimensions.}
\end{defn}


Note that there exists a natural injection, called \emph{homogenization}, from the set of polynomials in $K[x, y]$ and the set of homogeneous polynomials in $K[X, Y, Z]$ defined by $f(x, y) \mapsto Z^{\deg f}f(X/Z, Y/Z)$.
Because of this, we'll denote by $\mathfrak C$ both $\mathfrak C_f$ and $\mathfrak C_F$, where the usage is clear by context.

\begin{defn}[Rational points, \cite{Wal00}] %Not word-by-word
	Let $\mathfrak C$ be a projective curve defined by $F(X, Y, Z) = 0$, where $F \in K[X, Y, Z]$ is a homogeneous polynomial.
	A point $(X_0 : Y_0 : Z_0) \in \mathbb{P}^2(L)$ is called a \emph{$L$-rational point} on $\mathfrak C$ if $F(X_0, Y_0, Z_0) = 0$.
	Additionally, if $(X_0 : Y_0 : Z_0)$ is also in $\mathbb{P}^2(K)$, we call it a \emph{$K$-rational point} on $\mathfrak C$, or simply a \emph{rational point} on $\mathfrak C$.
\end{defn}

\begin{exmp*}
	Consider the curve $\mathfrak C$ defined by $X^2 + Y^2 + Z^2 = 0$ over $\mathbb{F}_7$.
	The point $(1 : 2 : 3)$ is a $\mathbb{F}_7$-rational point on $\mathfrak C$.
	Note that the polynomial $x^2 - 3$ is irreducible over $\mathbb{F}_7$, but there it has a root $\alpha$ in its algebraic closure, $\overline{\mathbb{F}_7}$. The projective point $(1 :  \alpha : \alpha)$ is a $\overline{\mathbb{F}_7}$-rational point on $\mathfrak C$.
\end{exmp*}


\begin{thm}[Bezout's theorem, \cite{Wal00}] %Not word-by-word
	If $F, G \in K[X, Y, Z]$ are homogeneous polynomials and there is no homogeneous polynomial in $K[X, Y, Z]$ that divides both $F$ and $G$, then their curves $\mathfrak C_F$ and $\mathfrak C_G$ intersect in exactly $\deg(F) \cdot \deg(G)$ $L$-rational points, counting multiplicity.
\end{thm}

%\begin{defn}[Rational function]
%	A rational function $f$ over $K$ is the ratio $g/h$ of polynomials $g, h \in K[x, y]$.
%	A homogeneous rational function $F$ over a $K$ is the ratio $G/H$ of homogeneous polynomials $G, H \in K[x, y]$.
%
%	After cancelling common roots of $g, h$, the roots of $g$ are called \emph{zeros} of $f$ and the roots of $h$ are called the \emph{poles} of $f$.  
%	We say $f$ has order $n$ at $P$ if $P$ is a zero of multiplicity $n$; order $-n$ if $P$ is a pole with multiplicity $n$; and order $0$, otherwise.
%\end{defn}

\begin{defn}[Rational function on a curve]
	Let $F \in K[X, Y, Z]$ be a homogeneous polynomial and let $\mathfrak C = \mathfrak C_F$.
	If $G, H \in K[X, Y, Z]$ are homogeneous polynomials of equal degree, then the fraction $G/H$ is called a \emph{rational function} on $\mathfrak C$.
	Fractions $G/H$ and $G'/H'$ define the same rational function if $G'H - GH'$ is a multiple of $F$, i.e., if  $G'H - GH'$ vanishes at all $L$-rational points on $\mathfrak C$.
\end{defn}

	Note that it makes sense to evaluate a rational function on a projective point, since \[
		\frac{G(X_0, Y_0, Z_0)}{H(X_0, Y_0, Z_0)} = \frac{\lambda^d G(X_0, Y_0, Z_0)}{\lambda^d H(X_0, Y_0, Z_0)} = \frac{G(\lambda X_0, \lambda Y_0, \lambda Z_0)}{H(\lambda X_0, \lambda Y_0, \lambda Z_0)},
	\]
	i.e., the result is the same regardless of the representation of a given projective point.

	After cancelling common factors of $G$ and $H$, a $L$-rational point $P$ on $\mathfrak C$ is called a \emph{zero} of $G/H$ whenever $G$ vanishes at $P$, and is called a \emph{poles} of $G/H$ whenever $H$ vanishes at $P$.  
	We say $G/H$ has order $n$ at $P$ if $P$ is a zero of multiplicity $n$; order $-n$ if $P$ is a pole with multiplicity $n$; and order $0$, otherwise.

\begin{defn}[Divisor] 
%	Let $\overline{\mathbb{F}_q}$ be the algebraic closure of $\mathbb{F}_q$.
%	Let $\mathfrak C$ be a projective plane curve defined over $\mathbb{F}_q$.
	Let $\mathfrak C$ be a projective plane curve defined over $K$.
	A \emph{divisor} $D$ on $\mathfrak C$ is a formal finite sum of the form $D = \sum a_P P$, where $P$ varies over the $L$-rational points on $\mathfrak C$, $a_P$ is an integer and $a_P = 0$ for all but finitely many points $P$.
	The \emph{degree} of $D$ is $\sum a_P$.
	The \emph{support} of $D$, denoted by $\Supp D$, is the set of $L$-rational points $P$ on $\mathfrak C$ that satisfy $a_P \neq 0$.
\end{defn}

%\begin{defn}[Field of rational functions on a curve, \cite{Wal00}] %Copied
%	Let $F$ be the polynomial which defines the nonsingular projective plane curve $\mathfrak C$ over the field $\mathbb{F}_q$. The \emph{field of rational functions on $C$} is
%	\begin{equation}
%		\mathbb{F}_q(\mathfrak C) := \left(\left\{\frac{G}{H}\, \bigg|\,\, \makecell{G, H \in \mathbb{F}_q[X, Y, Z]\\\text{are homogeneous}\\\text{of the same degree}} \right\} \cup \{0\}\right) / \sim
%	\end{equation}
%	where $g/h \sim g'/h'$ if, and only if, $gh' - g'h \in \langle F \rangle \subset \mathbb{F}_q[X, Y, Z]$.
%	\todo{Redefine it in a way that actually shows it is pretty simple.}
%\end{defn}

\begin{defn}[\cite{LS87}]
	If $D = \sum n_p P$, then define $\mathcal L(D)$ as the vector space formed by all homogeneous rational functions $f$ such that the order of $f$ at each $L$-rational point $P$ on $\mathfrak C$ is greater or equal to $-n_P$ (plus the zero polynomial).
\end{defn}

\begin{prop}\label{prop:degDlt0}
	If $\deg D < 0$, then $\mathcal L(D)$ only contains the zero polynomial.
\end{prop}

\begin{proof}
	Bezout's theorem implies that $\mathfrak C$ and $\mathfrak C_G$ intersect in the same number of points, counting multiplicity, than $\mathfrak C$ and $\mathfrak C_H$. Therefore, the number of zeros of a function is the same as the number of poles, counting multiplicity. Thus, the sum of the orders of $G/H$ at the $L$-rational points on $\mathfrak C$ must be zero. On the other hand, if $G/H \in \mathcal L(D)$, then the sum of the orders of $G/H$ at the $L$-rational points on $\mathfrak C$ must be $\ge - \deg(D)$.

	Thus, if $\deg(D) < 0$, then a rational function $G/H$ cannot be in $\mathcal L(D)$.
\end{proof}

For our study, an important theorem from Algebraic Geometry that will be useful in Section \ref{s:grs} when creating new types of codes is the following one:
\begin{thm}[Riemann--Roch Theorem, \cite{Wal00}]
	Let $\mathfrak C$ be a nonsingular projective curve of genus%
	\footnote{For the reader that is not familiar with genus, it is enough to know that it is an integer that can be calculated for any given curve.}
	$g$ defined over the field $\mathbb{F}_q$ and let $D$ be a divisor on $\mathfrak C$.
	Then
	\begin{equation}
		\dim \mathcal L(D) \ge \deg D + 1 - g,
	\end{equation}
	with equality holding if $\deg D > 2g - 2$.
\end{thm}

%Note that, if the degree of $D$ is negative, then
%\begin{equation}
%	\mathcal L(D) = \{0\}.
%\end{equation}
%Additionally, the \emph{Riemann--Rock theorem} implies that the dimension of the vector space $\mathcal L(D)$ is 
%\begin{equation}
%	 \dim\mathcal L(D) \ge \deg D - g + 1,
%\end{equation}
%where $g$ denotes the genus of $\mathfrak C$ (for the reader who is not familiar with the term genus, it suffices to know that it is an integer which can be calculated for any given curve).

\section{Generalized Reed--Solomon codes} \label{s:grs}
Let $\mathbb{P}^1(\mathbb{F}_q)$ denote the projective line over $\mathbb{F}_q = \{\alpha_1, \dots, \alpha_q\}$.
We will write $(a : b)$ to denote the projective point corresponding to the $1$-dimensional vector space containing $(a, b)$.
The points on $\mathbb{P}^1(\mathbb{F}_q)$ are the points
\begin{equation}
	P_i = (\alpha_i : 1), \quad 1 \le i \le q, \qquad \text{and} \qquad 
	P_\infty = (1 : 0).
\end{equation}

Consider the divisor $D = (k-1)P_\infty$ and the associated vector space $\mathcal L(D)$, which can be seen as the set of two-variable homogeneous rational functions which have a pole of order less than $k$ in the point $P_\infty$.

Note that, if a polynomial $f(x) \in \mathcal L_k$ has degree $d < k$, then the rational function \[
	F(X, Y) = \frac{Y^d f(X/Y)}{Y^d}
\]
has $P_\infty$ as its only pole, with order at most $k-1$. Thus, $F(X, Y) \in \mathcal L(D)$ and $f(\alpha_i) = F(P_i)$ for any $\alpha_i$.

Then, we can rewrite the Reed--Solomon code from Definition \ref{defn:rs_codes} as
\begin{equation}
	RS_q(n, k) = \left\{\big(f(P_1), f(P_2), \dots, f(P_n)\big) \mid f \in \mathcal L(D)\right\}.
\end{equation}

Here, we are evaluating the function $f$ in the points of the projective line.
We shall generalize the idea by changing the projective line to an arbitrary projective curve, and allowing other divisors.

\begin{defn}[Algebraic geometric codes, \cite{LS87, Wal00}] %Copied from Wal00
	Let $\mathfrak C$ be an irreducible nonsingular%
	\footnote{These adjectives just make sure $\mathfrak C$ does not have some weird behavior%
	%, e.g., $\mathfrak C$ has no double root%
	. The reader should not worry much about them.}
%%	\todo{Provide an explanation for why the `irreducible nonsingular' is there, but not a thing that the reader should worry. I think the key is some sort of multiplicity.}
	projective plane curve, let $D$ be a divisor on $\mathfrak C$ and let $\mathcal P = \{P_1, P_2, \dots, P_n\}$ be a set of rational points on $\mathfrak C$.
	Assume $\mathcal P$ and $\Supp D$ have no points in common, thus, no $P_i$ can be a pole of any $f \in \mathcal L(D)$, and $f(P_i)$ is well-defined for any $f \in \mathcal L(D)$ and any $P_i \in \mathcal P$.
	Then, the \emph{algebraic geometric code} associated to $\mathfrak C, \mathcal P$ and $D$ is
	\begin{equation}
		C(\mathfrak C, \mathcal P, D) := \{(f(P_1), \dots, f(P_n)) \mid f \in L(D)\} \subset \mathbb{F}_q^n.
	\end{equation}
\end{defn}

\begin{thm}[\cite{Wal00}]
	Let $\mathfrak C, \mathcal P, D$ be as above. Let $g$ denote the genus of $\mathfrak C$. Suppose $D$ satisfies $2g - 2 < \deg D < n$. Then the algebraic geometric code $C(\mathfrak C, \mathcal P, D)$ is linear of:
	\begin{enumerate}[label = \textbullet, itemsep = 0pt]
		\item length $n$,
		\item dimension $\deg D + 1 - g$,
		\item minimum distance $d \ge n - \deg D$.
	\end{enumerate}
\end{thm}

\begin{proof}
	Since $C(\mathfrak C, \mathcal P, D)$ is a subset of $\mathbb{F}_q^n$, it is a code of length $n$.

	Let $\psi: \mathcal L(D) \to \mathbb{F}_q^n$ be defined by $f \mapsto (f(P_1), \dots, f(P_n))$. This is a linear transformation, and $\mathcal L(D)$ is a vector space, thus the image of this function, which is precisely $C(\mathfrak C, \mathcal P, D)$, is also a vector space; thus it is a linear code.

	Futhermore, we shall prove that $\psi$ is injective.
	If $\psi(f) = \psi(g)$, then $\psi(f - g) = \vec 0$.
	Let $h := f-g$.
	We have $h(P_1) = h(P_2) = \dots = h(P_n) = 0$, so every $P_i$ is a zero of $h$, and finally $a_{P_i} \ge 1$, for all $1 \le i \le n$.
	Therefore, $h \in \mathcal L(D - P_1 - \cdots - P_n)$.
	Since $\deg D < n$, we have $\deg(D - P_1 - \cdots - P_n) < 0$ and, consequenly, $h$ must be the zero polynomial by Proposition \ref{prop:degDlt0}.
	Therefore, $f = g$, finishing the proof that $\psi$ is injective.
	Since $\psi$ is a injective linear transformation, the dimension of the domain and of the image are the same.
	Therefore, the dimension of $C(\mathfrak C, \mathcal P, D)$ is $k = \dim \mathcal L(D)$, which evaluates to $\deg D + 1 - g$ by the Riemann-Roch Theorem.

	Finally, let $f \neq g$ be polynomials such that $\psi(f)$ and $\psi(g)$ yield the minimum distance $d$.
	Then, $\psi(f) - \psi(g) = \psi(f-g)$ and $\psi(g) - \psi(g) = \vec 0$ also yield the same distance $d$.
	Therefore, if we define $h = f - g$, exactly  $d$ coordinates of $\psi(h)$ are nonzero.
	Without loss of generality, say that $h(P_{d+1}) = \cdots = h(P_n) = 0$.
	Similarly as in the previous paragraph, this implies $h \in \mathcal L(D - P_{d+1} - \cdots - P_n)$.
	Since $h$ is not the zero polynomial, Proposition \ref{prop:degDlt0} implies $\deg(D - P_{d+1} - \cdots - P_n) = \deg D - (n-d) \ge 0$, i.e., $d \ge n - \deg D$.
\end{proof}

This theorem is an interesting result! We want large $R$ and $\delta$, and these codes yield $k + d \ge n + 1 - g$, i.e., 
	\begin{equation}
		\label{eq:g/n} R + \delta \ge 1 + 1/n - g/n,
	\end{equation}
where $n$ is the number of rational points of a curve $\mathfrak C$, with genus $g$.
Futhermore, all the work done can be applied for algebraic curves in higher dimensional projective spaces.

\section{Final thoughts}\label{s:tvz}

On equation \eqref{eq:g/n}, we observe that good algebraic geometric codes are generated by curves with a small ratio between $g$ and $n$.
On \cite{TVZ82},\todo{Write one or two more sentences about this article.} the authors present a sequence of such curves, with $g/n$ large enough to create a better bound than the Gilbert--Varshamov one.

\printbibliography

\end{document}

% !TEX TS-program = pdflatexmk
\documentclass[11pt, letterpaper]{amsart}
%\usepackage{amsaddr}

\usepackage[isbn=false,url=false,doi=false,style=alphabetic]{biblatex} 
\addbibresource{sample.bib}

\usepackage{hyperref}

\usepackage{enumitem}
\setlist{itemsep=.5em, parsep = .5em} % < Usually a good idea

\title{Proposal for Term Paper: Bounds on Coding Theory from Algebraic Geometry}
\author{Guilherme Zeus Dantas e Moura}
%\address{Haverford College}
%\email{gdantasemo@haverford.edu}
%\date{Spring, 2021. Last change in \today}
\date{March 19, 2021}


\begin{document}

\maketitle

%\begin{abstract}    
%\end{abstract} 

\section{Proposal of Topic}
    Coding theory is concerned with finding efficient ways to encode a message so that one may discover errors in the message --- and perhaps even correct. In algebraic coding theory, we study efficient codes generated from algebraic geometric methods.
    
    In my proposed paper, I plan on constructing the Reed-Solomon codes, generalizing them using projective curves, and understand the results from \cite{TVZ82} on finding a bound better than the well-known Gilbert--Varshamov bound.
    
\section{Outline}

\subsection{Coding theory} \label{ss:codingtheory}
First, we need to define what a \emph{code} is, as well as what does it mean for a code to be good. We shall define length $n$, dimension $k$, minimum distance $d$, code rate $R = n/k$, and relative minimum distance $\delta = d/n$.

\subsection{Singleton bound and a promising example} \label{ss:singleton}
We shall state and prove the Singleton Bound for linear codes (state and maybe prove for non-linear codes) and define maximum distance separable (MDS) codes. Then, define Reed--Solomon codes and show they meet the Singleton Bound, but have a limited length in comparison to the alphabet size.

\subsection{Generalized Reed-Solomon codes} \label{ss:grs}
We shall redefine the Reed--Solomon codes using language related to a projective line. There is a way to replace the ``projective line'' with a ``projective plane curve'' and create other codes, called \emph{Generalized Reed-Solomon codes} or simply \emph{algebraic geometric codes}. We want large $R$ and $\delta$, and these codes yield \begin{equation}\label{eq:g/n} R + \delta \ge 1 + 1/n - g/n,\end{equation} where $n$ is the number of rational points of a curve $X$, with genus $g$.

\subsection{Final thoughts} \label{ss:tvz} On equation \eqref{eq:g/n}, we observe that good algebraic geometric codes are generated by curves with a large ratio between $n$ and $g$. On \cite{TVZ82}, the authors present a sequence of such curves, with $n/g$ large enough to create a better bound than the Gilbert--Varshamov one.

\section{Annotated Bibliography}

\begin{enumerate}
    \item[\textbf{\cite{TVZ82}}] \fullcite{TVZ82}.

    My proposed paper is aimed towards understanding the results of this article, as we see in subsection \ref{ss:tvz}. However, its language is rather complicated and heavy on algebraic geometry, so I will need to use other sources to understand it.
    
    \item[\textbf{\cite{TVN07}}] \fullcite{TVN07}.
    
    This is a book written by the authors of the previous article, that starts Algebraic Coding Theory from the scratch; thus it is an amazing resource to understand the language used in \cite{TVZ82}. However, it is quite long and has a lot of information not directly related to \cite{TVZ82}; so I'll primarily use this source to search for definitions and details about terms I encounter elsewhere.
    
    \item[\textbf{\cite{Wal00}}] \fullcite{Wal00}.
    
    Aimed for undergraduates, this book explains definitions and motivations from coding theory (which correspond to subsections \ref{ss:codingtheory} and \ref{ss:singleton}); explains facts from algebraic geometry; describes the algebraic geometric codes; and discusses the results from \cite{TVZ82}.
    
    \item[\textbf{\cite{LS87}}] \fullcite{LS87}.
    
    This is an expository article that aims to simplify the methods from algebraic geometry used in \cite{TVZ82}. Their approach is similar to the one found in \cite{Wal00}, but it is much shorter and concise. It will be a great complement to the other sources. 
    
\end{enumerate}


\end{document}